\documentclass[11pt]{article}
 
\usepackage{mathtext}         % если нужны русские буквы в формулах (не обязательно)
\usepackage[T2A]{fontenc}     % внутренняя T2A кодировка TeX
\usepackage[russian]{babel}   % включение переносов
\usepackage[utf8]{inputenc}
\usepackage[margin=0.5in]{geometry}
\pagestyle{empty} % нумерация выкл.
\addtolength{\textheight}{1.75in}
\usepackage{hyperref}
\usepackage{longtable}
\usepackage{color}
\usepackage{setspace}
\definecolor{gray}{rgb}{0.4,0.4,0.4}

 

\setstretch{0.8}

\begin{document}
 
\noindent {\sffamily{\Huge{\textbf{   Трубицын Юрий Алексеевич   }}}}%
 
\vspace{0.5em}
 
\noindent +7 926 048 10 43  |   \href{mailto:trubicyn.yura@gmail.com}{trubicyn.yura@gmail.com}  |  Москва, Россия  | 3 октября, 1995
 
\vspace{0.5em}
 
\hrule
 
\vspace{1.5em}

\noindent {\textbf{Желаемая должность}}: Программист-разработчик, исследователь. 
 
\vspace{1em} 
 
\noindent {\textbf{Цель}}: Работать/стажироваться в стабильной компании с опытными разработчиками. 
 
\vspace{1em}
 
\noindent {\textbf{О себе:}}
\begin{itemize}
\item Начало карьеры, студент;
\item Аналитическое мышление, быстрообучаемость;
\item Желание решать сложные задачи;
\item Стремление узнавать новое;
\item Языки - Английский (читаю профессиональную литературу).
\end{itemize}

\vspace{1em}

\noindent {\textbf{Интересы:}}
\begin{itemize}
\item Информационные технологии, Интернет;
\item Физика;
\item Наука, Образование.
\end{itemize}

\vspace{1em} 
 
\noindent {\textbf{Образование:}}
 
\begin{longtable} {l | p{1.0\textwidth}}
 
2013 - 2017 & {\textbf{«\href{http://cs.msu.ru}{МГУ им. Ломоносова, ф-т ВМК}»} \newline
 \textbf{\href{http://mk.cs.msu.ru}{кафедра Математической кибернетики}} \newline
 \textbf{средний балл: 4.68}}\\  

\end{longtable}

\noindent {\textbf{ Языки программирования, ОС и технологии:}}

\begin{longtable} {l | p{0.85\textwidth}}
 {\textbf{Отличный}} & C++, C, Pascal \\
 {\textbf{Неплохой}} & Python, Unix, Java \\
 {\textbf{Базовый}}  & Assembler, Git, Verilog, Matlab \\
\end{longtable} 


\noindent {\textbf{ Опыт:}}

\begin{longtable} {l | p{0.7\textwidth}}
 {\textbf{2014 г.}} & Стажер в математическом лагере(олимпиадная математика 6-11 класс) \\
 {\textbf{2014 г. - ...}} & Репетитор по математике, информатике и программированию(школьники и студенты младших курсов технических вузов) \\
 {\textbf{2015 г. - ...}} & Частная практика(пишу различные небольшие приложения для частных лиц) \\
 {\textbf{июнь - декабрь, 2016 г.}} & Андроид-разработчик в компании <<Sprinwood>>\\
\end{longtable}
 
\end{document}
